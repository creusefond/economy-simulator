\title{Price derivation on fixed supply}

During the phase of the model where individuals buy products, the problem of the price arises.
At what price are the goods exchanged ?
How are the products consequently shared between individuals ?

To solve this step of the model, we use the following assumptions :
\begin{enumerate}
    \item Each individual will spend his entire allowance to maximize his utility.
    \item All the products will be purchased.
\end{enumerate}

The strategy exposed here is the following :
\begin{enumerate}
    \item Define the utility that an individual gains from owning a product.
    \item Derive the quantities of products that an individual would purchase given a set of prices.
    \item Determine the price of each product, and therefore the quantity actually purchased by each individual.
\end{enumerate}

\subtitle{Utility}
Each individual $u\in U$ has a global utility function $utility(u, q_{u, A}, q_{u, B}, \dots)$ that depends on the quantities of each product that is owned by the individual.
We assume that the global utility function of an user  is a linear combination of the utility

A standard assumption is to consider diminishing returns function : an individual benefits less and less of additional poducts.
This 

\subtitle{Determination of quantities that each individual want to purchase}
