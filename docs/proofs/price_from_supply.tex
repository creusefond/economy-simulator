\section{Price derivation on fixed supply}

During the phase of the model where individuals buy products, the problem of the price arises.
At what price are the goods exchanged ?
How are the products consequently shared between individuals ?

To solve this step of the model, we use the following assumptions :
\begin{enumerate}
    \item Each individual will spend his entire allowance to maximize his utility.
    \item All the products available will be purchased.
    \item Utility gained from a product is independant to the utility of other products.
    \item Individuals can purchase fractions of a product.
    \item Prices are global.
\end{enumerate}

The strategy exposed here is the following :
\begin{enumerate}
    \item Define the utility that an individual gains from owning a product.
    \item Derive the quantities of products that an individual would purchase given a set of prices.
    \item Determine the price of each product, and therefore the quantity actually purchased by each individual.
\end{enumerate}

\subsection{Utility}

\subsubsection{Global utility}
Each individual $u\in U$ has a global utility function $utility(u, q_{u, A}, q_{u, B}, \dots)$ that depends on the quantities $q_{u, p}$ of each product $p\in P$ that is owned by the individual.
We assume that $utility$ is a linear combination of the utility of the multiple products :
\begin{equation}
    \label{eq:global_utility}
    utility(u, q_{u, A}, q_{u, B}, \dots) = \sum_{p\in P} w_{u, p}\times utility_p(q_{u, p})
\end{equation}

For the simulation purpose, we considered all utility weights $w_{u, p}$ to be derived from a model parameter at product-level.
These parameters will allow the simulation to model different products with different global utilities.
For each user, we draw a random number for each product that is weighted by the model parameter.
Each utility weight $w_{u, p}$ correspond to this number, normalised so that :
\begin{equation}
    \label{eq:weight_normalisation}
    \sum_{p\in P}w_{u, p} = 1
\end{equation}

\subsubsection{Product utility}
We consider that each user has a utility function for each product.
Right now, we assume that all product utility functions are the same, but we would like to move from this assumption in the future.

We will only consider concave product utility functions.
This is implied by the standard assumption of diminishing marginal utility : an individual benefits less and less of additional poducts.
Note that it implies that the global utility is concave as well.

The first product utility function is :
\begin{equation}
    utility\_ln(x) = ln(x + 1)
\end{equation}

Notes on this function :
\begin{itemize}
    \item it is strictly increasing on $[0; +\infty[$.
    \item it is normalised so that $utility\_ln(0) = 0$ and we avoided the lack of definition of $ln$ on $0$
\end{itemize}

The second product utility function is :
\begin{equation}
    utility\_quadratic(x) = (a - x)^2 + a^2
\end{equation}

Notes on this function :
\begin{itemize}
    \item it has a global maximum on a constant quantity $a$
    \item its derivative is linear, so it assumes a marginal utility that is constantly decreasing
    \item it is normalised so that $utility\_quadratic(0) = 0$
\end{itemize}

To simplify, we will first present the reasonning and result using $utility\_ln$.
Then, we will provide the same results using $utility\_quadratic$.

\subsection{Determination of quantities that each individual want to purchase}

Let's assume that each individual $u\in U$ has an amount $m_u$ of money available.
Let's also assume that each product $p\in P$ is attributed a price $price_p$.
How much of each item will the individual purchase in order to maximize his utility ?

The problem can be stated as the minimisation problem of a convex function $f(q)$ where $q\in \mathbb{R}_{\ge0}^{|P|}$ are the quantities of the different products.
This is the inverse of the global utility eq.\eqref{eq:global_utility} :
\begin{equation*}
    f(q) = (-1)\times \sum_{p\in P} w_{u, p}\times utility\_ln(q_{u, p})
\end{equation*}

% TODO : how to do with quadratic ? There will be times when they don't want to spend more. Need to implement financial products.
And the problem statement is the following, using a constraint fonction $f_c(q)$ to represent the constraint that all the money need to be spent :
\begin{equation}
\begin{split}
    min f(q) \\
    f_c(q) = m_u - \sum_{p\in P}q_{u, p}\times price_p = 0
\end{split}
\end{equation}

We use lagrangian multipliers to get a set of equations that we will use to find the minimum.
\begin{equation*}
\begin{cases}
\nabla f(q) + \lambda \nabla f_c(q) & = (0, 0, \dots, 0) \\
f_c(q) & = 0
\end{cases}
\end{equation*}

\begin{equation*}
\Leftrightarrow 
\begin{cases}
\forall p\in P, - w_{u,p}\times utility\_ln'(q_{u, p}) - \lambda \times price_p & = 0 \\
m_u - \sum_{p\in P}q_{u, p}\times price_p & = 0
\end{cases}
\end{equation*}

\begin{equation*}
\Leftrightarrow 
\begin{cases}
\forall p\in P, \dfrac{w_{u,p}}{q_{u,p} + 1} + \lambda \times price_p = 0 \Leftrightarrow q_{u,p} = \dfrac{- w_{u, p}}{\lambda \times price_p} - 1 \\
m_u - \sum_{p\in P}(- \dfrac{w_{u, p}}{\lambda \times price_p} - 1)\times price_p = 0 \Leftrightarrow \lambda = \dfrac{- \sum w_{u,p}}{m_u + \sum price_p}
\end{cases}
\end{equation*}

Combining with eq.\eqref{eq:weight_normalisation}, we find the quantities that optimise the utility :
\begin{equation}
    \forall p\in P, q_{u,p} = \dfrac{(m_u + \sum price_p) \times w_{u, p}}{price_p} - 1
\end{equation}